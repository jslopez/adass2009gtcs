%ADASS_PROCEEDINGS_FORM%%%%%%%%%%%%%%%%%%%%%%%%%%%%%%%%%
%
% TEMPLATE.TEX -- ADASS Conference Proceedings template.
%
% Use this template to create your proceedings paper in LaTeX format
% by following the instructions given below.  Much of the input will
% be enclosed by braces (i.e., { }).  The percent sign, "%", denotes
% the start of a comment; text after it will be ignored by LaTeX.  
% You might also notice in some of the examples below the use of "\ "
% after a period; this prevents LaTeX from interpreting the period as
% the end of a sentence and putting extra space after it.  
% 
% You should check your paper by processing it with LaTeX.  For
% details about how to run LaTeX as well as how to print out the User
% Guide, consult the README file.  You should also consult the sample
% LaTeX papers, sample1.tex and sample2.tex, for examples of including
% figures, html links, special symbols, and other advanced features.
%
%%%%%%%%%%%%%%%%%%%%%%%%%%%%%%%%%%%%%%%%%%%%%%%%%%
% Note that the primary style file is that from the ASP Conf. Series; ADASS style 
% elements are included by an additional \usepackage. You may use other 
% _standard_ packages if needed, such as lscape, psfig, epsf, and graphicx, 
% although these packages may already be installed on your system. 
%
\documentclass[11pt,twoside]{article}  % Leave intact
\usepackage{asp2006}
\usepackage{adassconf}

% Set counters for footnotes and sectioning, which is needed when 
% constructing the full volume of all papers. 
% DO NOT DELETE. 
\setcounter{equation}{0}
\setcounter{figure}{0}
\setcounter{footnote}{0}
\setcounter{section}{0}
\setcounter{table}{0}

\begin{document}   % Leave intact

%-----------------------------------------------------------------------
%			    Paper ID Code
%-----------------------------------------------------------------------
% Enter the proper paper identification code.  The ID code for your paper 
% is the session number associated with your presentation as published 
% in the official conference proceedings.  You can find this number by 
% locating your abstract in the printed proceedings that you received 
% at the meeting, or on-line at the conference web site.
%
% This identifier will not appear in your paper; however, it allows different
% papers in the proceedings to cross-reference each other.  Note that
% you should only have one \paperID, and it should not include a
% trailing period.
%
% EXAMPLE: \paperID{O4.1}
% EXAMPLE: \paperID{P2.7}

\paperID{43}

%-----------------------------------------------------------------------
%		            Paper Title 
%-----------------------------------------------------------------------
% Enter the title of the paper.
%
% EXAMPLE: \title{A Breakthrough in Astronomical Software Development}

\title{A Reference Architecture Specification of a Generic Telescope Control System}

%-----------------------------------------------------------------------
%          Short Title & Author list for page headers
%-----------------------------------------------------------------------
% Please supply the author list and the title (abbreviated if necessary) as 
% arguments to \markboth.
%
% The author last names for the page header must appear in one of 
% these formats:
%
% EXAMPLES:
%     LASTNAME
%     LASTNAME1 and LASTNAME2
%     LASTNAME1, LASTNAME2, and LASTNAME3
%     LASTNAME et al.
%
% Use the "et al." form in the case of four or more authors.
%
% If the title is too long to fit in the header, shorten it: 
%
% EXAMPLE: change
%    Rapid Development for Distributed Computing, with Implications for the Virtual Observatory
% to:
%    Rapid Development for Distributed Computing

\markboth{}{}

%-----------------------------------------------------------------------
%		          Authors of Paper
%-----------------------------------------------------------------------
% Enter the authors followed by their affiliations.  The \author and
% \affil commands may appear multiple times as necessary.  List each
% author by giving the first name or initials first followed by the
% last name. Do not include street addresses and postal codes, but 
% do include the country name or abbreviation. 
%
% If the list of authors is lengthy and there are several institutional 
% affiliations, you can save space by using the \altaffilmark and \altaffiltext 
% commands in place of the \affil command.
%
% EXAMPLE: 
%      \author{Raymond Plante, Doug Roberts, 
%                  R.\ M.\ Crutcher\altaffilmark{1}}
%      \affil{National Center for Supercomputing Applications, 
%                 University of Illinois, Urbana, IL, USA}
%      \author{Tom Troland}
%      \affil{University of Kentucky, Lexington, KY, USA}
%
%      \altaffiltext{1}{Astronomy Department, UIUC}
%
% In this example, the first three authors, "Plante", "Roberts", and
% "Crutcher" are affiliated with "NCSA".  "Crutcher" has an alternate 
% affiliation with the "Astronomy Department".  The fourth author,
% "Troland", is affiliated with "University of Kentucky"

\author{Joao S. L\'opez, Rodrigo J. Tobar, Tomas Staig, Daniel A. Bustamante, Camilo E. Menay, Horst H. von Brand}
\affil{Universidad T\'ecnica Federico Santa Mar\'ia, Valpara\'iso, Chile}
\author{Mauricio A. Araya}
\affil{Institut National de Recherche en Informatique et Automatique, Nancy, France}

%----------------------------------------------------------------------
%			 Contact Information
%-----------------------------------------------------------------------
% This information will not appear in the paper but will be used by
% the editors in case you need to be contacted concerning your
% submission.  Enter your name as the contact along with your email
% address.
% 
% EXAMPLE:  \contact{Dennis Crabtree}
%           \email{crabtree@cfht.hawaii.edu}

\contact{Joao S. Lopez}
\email{jslopez@csrg.inf.utfsm.cl}

%-----------------------------------------------------------------------
%		      Author Index Specification
%-----------------------------------------------------------------------
% Specify how each author name should appear in the author index.  The 
% \paindex{ } should be used to indicate the primary author, and the
% \aindex for all other co-authors.  You MUST use the following
% syntax: 
%
% SYNTAX:  \aindex{Lastname, F.~M.}
% 
% where F is the first initial and M is the second initial (if used). Please 
% ensure that there are no extraneous spaces anywhere within the command 
% argument. This guarantees that authors that appear in multiple papers
% will appear only once in the author index. Authors must be listed in the order
% of the \paindex and \aindex commmands.
%
% EXAMPLE: \paindex{Crabtree, D.}
%          \aindex{Manset, N.}        
%          \aindex{Veillet, C.}        

\paindex{L\'opez, J.~S.}
\aindex{Tobar, R.~J.}     % Remove this line if there is only one author
\aindex{Staig, T.}
\aindex{Bustamante, D.~A.}
\aindex{Menay, C.~E.}
\aindex{Araya, M.~A.}
\aindex{von Brand, H.~H.}

%-----------------------------------------------------------------------
%			Subject Index keywords
%-----------------------------------------------------------------------
% Enter up to 6 keywords that are relevant to the topic of your paper.  These 
% will NOT be printed as part of your paper; however, they will guide the creation 
% of the subject index for the proceedings.  Please use entries from the
% standard list where possible, which can be found in the index for the 
% ADASS XVI proceedings. Separate topics from sub-topics with an exclamation 
% point (!). 
%
% EXAMPLE:  \keywords{astronomy!radio, computing!grid, data management!workflows, 
%     instrumentation!control}

\keywords{instrumentation!control, software!reuse, software!frameworks, computing!distributed}

%-----------------------------------------------------------------------
%			       Abstract
%-----------------------------------------------------------------------
% Type abstract in the space below.  Consult the User Guide and Latex
% Information file for a list of supported macros (e.g. for typesetting 
% special symbols). Do not leave a blank line between \begin{abstract} 
% and the start of your text.

\begin{abstract}          % Leave intact
A Telescope Control System (TCS) is a software responsible of controlling the
hardware that an astronomical observation needs. The automation and
sophistication of these observations has produced complex systems. Currently, a
TCS is compound by software components that interact with several users and
even with other systems and instruments.
Each observatory has successfully developed a wide spectrum of TCS solutions
for their telescopes. Regardless the mount, there are common patterns in the
software components that all these telescopes use. As almost every telescope is
custom designed, these patterns are reimplemented again and again for each
telescope. This is indeed an opportunity of reuse and collaboration.
The Generic Telescope Control System (gTCS) pretends to be a base distributed
framework for the development and deployment of the TCS of any telescope,
independent of its physical structure, the type of mount and instrumentation
that they use. This work presents an architecture specification explained
through two complementary approaches: the layers perspective and the deployment
perspective. The first approach defines a set of layers, one on the top of the
other, offering different levels of abstraction. Meanwhile the deployment
perspective intends to illustrate how the system could be deployed, focused on
the distributed nature of the devices.
\end{abstract}

%-----------------------------------------------------------------------
%			      Main Body
%-----------------------------------------------------------------------
% Place the text for the main body of the paper here.  You should use
% the \section command to label the various sections; use of
% \subsection is optional.  Significant words in section titles should
% be capitalized.  Sections and subsections will be numbered
% automatically. 
%
% EXAMPLE:  \section{Introduction}
%           ...
%           \subsection{Our View of the World}
%           ...
%           \section{A New Approach}
%
% It is recommended that you look at the sample paper sample2.tex
% for examples of formatting references, footnotes, figures, equations, 
% html links, lists, and other features.  

\section{The Generic TCS Problem}
\subsection{Overview}
Control systems have two basic entry points: the \textit{users} and the
\textit{devices}. In a TCS domain, users and devices are heterogeneous: users
with various levels of expertise, and devices with different protocols and
access levels. Therefore, to understand the problem we must identify the diverse
nature of users and devices.

\subsection{The Users}
Users can be classified by the \emph{usage} that they give to the system, or in
other words, to which \emph{profile} they belong. Profile examples:

\begin{itemize}
        \item \textbf{Observation Control}: The control of the observation
itself. Includes all the variables in the domain of the Astronomer, and all the
technical and specific details of the telescopes are hidden.
        \item \textbf{Calibration and Startup}: The automatic or manual process
of calibrating the telescope, also in high-level domain, but including the
specific details of the telescope.
        \item \textbf{Maintenance}: The low-level variables of the telescope,
with a detailed control of all the devices and software states.
        \item \textbf{Monitoring}: The summary of the telescope operations to
audit, check the behavior of the devices, etc. This may be a mixture of specific
logs with general information of the observation.
\end{itemize}

The problem is that several users could use more than one profile. Then,
building an application for each profile is not the most desired approach.
Therefore, the existing TCSs often build very complex user interfaces that have
all the information that the user \emph{may} need. This turns the
application unmanageable for unexperienced users. Fortunately, defining these
profiles helps to identify which is the scope of the TCS and select the features
that the system will need in a user-independent fashion.

\subsection{Devices Scope}
The devices of a telescope are diverse. Only in the axis control domain each
telescope mount/technology has different devices with different protocols and
configurations. Even if two telescopes has the same hardware, the firmware or
other vendor software could vary. If we add to the equation mirror control,
active optics and meteorologic stations, the set of devices turns unmanageable.
Therefore, defining the possible set of devices is not a practical approach. A
simpler approach is to group the devices into \emph{instruments} that do a
specific task.  In an observatory there are two general types of hardware
devices:

\begin{itemize}
        \item \textbf{Technical Instruments}: All instrument that does not
directly produce scientific data, such as the telescope, a meteorologic station,
the active optics, an autoguiding CCD, etc.
        \item \textbf{Scientific Instruments}: All instrument that produces
scientific data that the astronomer will use, such as the main CCD, a
spectrograph, an interferometer, filter wheels, etc.  \end{itemize}

The scope of a TCS is limited to \emph{technical instruments} and their devices.
Also, a TCS must provide all the interfaces to connect the software in charge of
managing the \emph{scientific instruments}.

\section{Proposed Architecture Specification}
\subsection{Layers perspective}
The \emph{layers perspective} presents different levels of abstraction (higher
layers offer higher abstractions). This perspective can be seen as the network
stack (such TCP/IP) where each layer offers services to the upper one, and the
upper one only uses the lower one.

This view allow us to have an idea of which are the different abstractions
that the system needs, and how to encapsulate this information in different
layers.

\begin{description}
        \item[Drivers] Responsible of the communication of the system with
                external entities.
        \item[Sensors/Actuators] Considering a device as a single monitor/control point
                associated with a unique property to be read/written,
                Sensors/Actuators provides software abstractions of physical
                devices, without considering any interaction with any other
                device.
        \item[Composite devices] It represents a full device used on the
                astronomy domain that can be composed by a given number of
                sensors and/or actuators.
        \item[Observation control] It offers the abstraction of a full equipped
                telescope. This is done by grouping different composite
                devices, and using them in an intelligent way, in order to
                obtain data, collaborate, and finally do control over the
                necessary composite devices.
\end{description}

\subsection{Deployment perspective}
The \emph{deployment perspective} intends to illustrate how the architecture can
be used in a distributed way, showing the geographical deployment of the
system, and the bus that communicates its components.

Each computer will execute different parts of the system and we can distribute one
layer through several computers. This is the capability of the system to have
its components spread over a network of computers, while working together.

The Information Service can be seen as a software component accessible through
any part part of the system. It has two main responsibilities: to know which
software components are available in the system and to offer the possibility to
retrieve information about the classes, interfaces, methods, parameters and
related information from all the components of the system.

\section{Reference technologies}
The Control System for an Amateur Telescope (Tobar et al. 2008) project, a TCS constructed over
ALMA Common Software (ACS) (Chiozzi et al. 2008, Chiozzi et al. 2006) was used as an initial approach to the problem and
reference architecture.

To exemplify the proposed reference, existing technologies that are implementing
some aspects of the proposal are analyzed.

\begin{description}
        \item[ALMA Common Software] If we consider the Container/Component (Sommer et al. 2004)
                model present on ACS, each composite device can be managed by an
                ACS Characteristic component. Through the use of states,
                exceptions and container lifecycle we are able to manage the
                lifecycle of the whole gTCS. Currently ACS uses CORBA, which is
                an example of a communication bus. The information service can
                be obtained through the Manager (responsible for the management
                of containers, components and clients) and the information
                provided by the Interface Repository (IR) and the Configuration
                Database (CDB).
        \item[VLT Common Software] The VLT control software (Chiozzi et al. 1995) uses software
                modules, sharing benefits of re-usability. These modules are a
                basic item for the detailed design, development and integration
                of software.  The software architecture is distributed over
                several workstations that provide high level and coordination
                services. The communication is based on a message system and a
                distributed hierarchical database.

        \item[Java Remote Method Invocation] The Java Remote Method Invocation
                (RMI) (Grosso 2002) is a Java approach to support a model of a distributed
                object application. It provides remote communication between
                programs written in Java. It allows applications to call
                methods located remotely, sharing resources and processing load
                across systems.
\end{description}

\acknowledgments
This work was supported by ALMA-CONICYT Fund project \#31060008
\textit{``Software Development for ALMA: Building Up Expertise to Meet ALMA
Software Requirements within a Chilean University''}, under development at
Universidad T\'ecnica Federico Santa Mar\'ia.

%-----------------------------------------------------------------------
%			      References
%-----------------------------------------------------------------------
% List your references below within the reference environment
% (i.e. between the \begin{references} and \end{references} tags).
% Each new reference should begin with a \reference command which sets
% up the proper indentation.  
%    NOTE: all citations in the text _must_ have a corresponding entry in 
%    the reference list, and all references must be cited in the text.
%
% Observe the following order when listing bibliographical 
% information for each reference:  author name(s), publication 
% year, journal name, volume, and page number for articles. 
% URLs to the reference may be given either in-line, or as a footnote. 
% Note that many journal names are available as macros; see
% the User Guide for a listing "macro-ized" journals. 
%
% EXAMPLES:  
% Reference to a Journal article:
%     \reference Cornwell, T.\ J.\ 1988, \aap, 202, 316
%
% Journal paper with more than 7 authors;
%     \reference Hanisch, R.\ et al.\ 2001, \aap, 376, 359
%
% Reference to an SPIE paper:
%     \reference Noordam, J.~E.\ 2004, Proc.\ SPIE, 5489, 817
%
% Reference to a contribution to a proceedings (not ADASS)
%     \reference Schmitz, M., Helou, G., Dubois, P., LaGue, C., Madore,B., Corwin, H.~G., Jr., 
%          \& Lesteven, S.\ 1995, in Information \& On-Line Data in Astronomy, 
%          ed.\ D.\ Egret \& M.~A.\ Albrecht (Dordrecht: Kluwer Academic Publishers), 259
%
% Reference to a paper in an earlier ADASS proceedings:
%     \reference Kantor, J., et al.\ 2007, \adassvii, 3
%
% Reference to a paper in the current ADASS:
%     \reference Hanisch, R.~J.\ 2008, \adassxvii, \paperref{O1.3}
% 
% Reference to a book:
%     \reference Jacobson, I.\ Booch, G., \& Rumbaugh, J.\ 1999, 
%            The Unified Software Development Process (Reading, MA: Addison-Wesley)
%
% Reference to a thesis:
%     \reference Gering, D.\ 1999, Master's Thesis, Massachusetts Institute of Technology
% 
% Reference to a purely on-line resource:
%     \reference Staveley-Smith, L.\ 2006, ATNF SKA Memo~6, http://www.atnf.csiro.au/ska
%
% Note the following tricks used in the example above:
%
%   o  \& is used to format an ampersand symbol (&).
%   o  \'e puts an accent agu over the letter e.  See the User Guide
%      and the sample files for details on formatting special
%      characters.  
%   o  "\ " after a period prevents LaTeX from interpreting the period 
%      as an end of a sentence.
%   o  \aj is a macro that expands to "Astron. J."  See the User Guide
%      for a full list of journal macros
%   o  \adassvii is a macro that expands to the full title, editor,
%      and publishing information for the ADASS VII conference
%      proceedings.  Such macros are defined for ADASS conferences I
%      through XVI.
%   o  When referencing a paper in the current volume, use the
%      \adassxvii and \paperref macros.  The argument to \paperref is
%      the paper ID code for the paper you are referencing.  See the 
%      note in the "Paper ID Code" section above for details on how to 
%      determine the paper ID code for the paper you reference.  
%
\begin{references}
\reference Tobar, R. et~al., ``An amateur telescope control system towards a generic
telescope control model'' in Proceedings of SPIE - Advanced Software and
Control for Astronomy 2008, (2008).

\reference Chiozzi, G. et~al., ``The ALMA Common Software: A developer friendly CORBA
based framework'' in Proceedings of SPIE - Advanced Software and Control
for Astronomy 2008, (2008).

\reference Chiozzi, G. et~al., ``Application development using the ALMA Common
Software'' in Proceedings of SPIE - Advanced Software and Control for
Astronomy 2006, (2006).

\reference Sommer, H. et~al., ``Container-component model and XML in ALMA ACS''
in Proceedings of SPIE - Advanced Software and Control for Astronomy 2004,
(2004).

\reference Chiozzi, G., ``An object-oriented event-driven architecture for the VLT
Telescope Control Software'' in Proceedings of ICALEPCS 1995
, (1995).

\reference Grosso, W., ``Java RMI'', O'Reilly \& Associates Inc, (2002).
\end{references}

% Do not place any material after the references section

\end{document}  % Leave intact
